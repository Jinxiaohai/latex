%#-*- coding:utf-8 -*-
\documentclass[11pt,UTF8,hyperref,openany]{ctexbook}
\usepackage{amsmath}             %%%%多种的公式环境和数学命令
\usepackage{amssymb}             %%%%数学符号生成命令
\usepackage{array}               %%%%数组和表格
\usepackage{booktabs}            %%%%水平的表格线
\usepackage{calc}                %%%%四则运算
\usepackage{caption}             %%%%插图和表格
% \usepackage{ctex}                %%%%中文字体
\usepackage{ctexcap}             %%%%中文字体和标题
\usepackage{color}
\usepackage{fancyhdr}            %%%%页眉页脚设置
\usepackage{graphicx}            %%%%插图
\usepackage{geometry}            %%%%版面尺寸控制
\geometry{left=2cm, right=2cm, top=2cm, bottom=2cm, head=2cm, foot=1cm}
% head=?cm, headmap=?cm
\usepackage{hyperref}            %%%%超链接
\usepackage{ifthen}              %%%%条件
\usepackage{longtable}           %%%%跨页表格
\usepackage{lineno}              %%%%行号控制
\usepackage{listings}            %%%%C++
\usepackage{multicol}            %%%%多栏
\usepackage{makeidx}             %%%%索引
\usepackage{ntheorem}            %%%%定理设置
\usepackage{paralist}            %%%%列表
\usepackage{tabularx}            %%%%表格的列宽
\usepackage{titlesec}            %%%%章节标题
\usepackage{fancyvrb}            %%%%抄录
\usepackage{fontspec}            %%%%字体
\usepackage{titletoc}            %%%%目录格式
\usepackage{xcolor}              %%%%颜色处理
\usepackage{xeCJK}               %%%%中日朝文字处理
%%%%%%%%%%%%%%%%%%%%%%%%%%%%%%%%%%%%%%%%%%%%%%%%%%%%%%%%%% 
% C++
\definecolor{keywordcolor}{rgb}{0, 0, 1}
\lstset{breaklines}
\lstset{extendedchars=false}
\lstset{language=C++, keywordstyle=\color{keywordcolor}\bfseries,
  basicstyle=\ttfamily,
  showstringspaces=false,
  captionpos=b
}

%%%%%%%%%%%%%%%%%%%%%%%%%%%%%%%%%%%%%%%%%%%%%%%%%%%%%%%%%% 
% newcommand
\newcommand{\mycmdA}{ }
\newcommand{\mycmdB}[1]{{\heiti #1}}
\newcommand{\mycmdC}[2]{$#1_1,#1_2,\dots,#1_#2$}


\author{xiaohai}
\date{1990/10/12}
\title{GDR experiment}

\begin{document}
% 页码的设置
\pagenumbering{Roman}

\maketitle

%%%%%%%%%%%%%%%%%%%%%%%%%%%%%%%%%%%%%%%%%%%%%%%%%%%%%%%%%% 
% 目录的设置
\titlecontents{chapter}[4em]{\addvspace{2.3mm}\bf}{%
  \contentslabel{4.0em}}{}{\titlerule*[5pt]{$\cdot$}\contentspage}
\titlecontents{section}[4em]{}{\contentslabel{2.5em}}{}{%
  \titlerule*[5pt]{$\cdot$}\contentspage}
\titlecontents{subsection}[7.2em]{}{\contentslabel{3.3em}}{}{%
  \titlerule*[5pt]{$\cdot$}\contentspage}
\tableofcontents

% 设置插图的目录
\listoffigures
% 设置表格的目录
\listoftables

%%%%%%%%%%%%%%%%%%%%%%%%%%%%%%%%%%%%%%%%%%%%%%%%%%%%%%%%%% 
\chapter{Class}
%%%%%%%%%%%%%%%%%%%%%%%%%%%%%%%%%%%%%%%%%%%%%%%%%%%%%%%%%% 
\pagenumbering{arabic}
% footnote
\renewcommand{\thefootnote}{}
\footnote{\heiti{ 作者简介}: 金小海,男}
\footnote{\heiti{ 现在单位}: 不正常人类研究中心}
% 将脚注的引用记数设为0.
\setcounter{footnote}{0}
\renewcommand{\thefootnote}{\arabic{footnote}}
\begin{verbatim}
Mod103_TModV830AC
Mod104_TModV785
Mod106_TModV785
Mod108_TModV785
Mod110_TModV785
Mod112_TModV785
Mod114_TModV785N
Mod116_TModV775N
Mod118_TModV775
Mod120_TModV775
\end{verbatim}



%%%%%%%%%%%%%%%%%%%%%%%%%%%%%%%%%%%%%%%%%%%%%%%%%%%%%%%%%% 
% 附录
\appendix
\chapter{C++程序}
\section{Ising model}
\section{Monter carlo method}

%%%%%%%%%%%%%%%%%%%%%%%%%%%%%%%%%%%%%%%%%%%%%%%%%%%%%%%%%% 
% % 链接的方式
% \noindent\url{xiaohaijin@outlook.com}\newline
% \href{http://192.168.1.119/self/index.html}{自己的http目录}\newline
% \href{http://10.10.11.64}{64}\newline
% % 页码链接
% 点我回到第一页\hyperpage{1}

% %%%%%%%%%%%%%%%%%%%%%%%%%%%%%%%%%%%%%%%%%%%%%%%%%%%%%%%%%% 
% \begin{lstlisting}
%   #include <iostream>
%   using namespace std;

%   int main(int argc, char** argv)
%   {
%     cout << "Hello Tex." << endl;
%     return 0;
%   }
% \end{lstlisting}

\end{document}
