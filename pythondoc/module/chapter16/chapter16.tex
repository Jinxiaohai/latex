%#-*- coding:utf-8 -*-
\chapter{Generic Operating System Services}
\section{os--Miscellaneous operating system interfaces}
该模块提供了方便的使用操作系统依赖的函数的方式。如果你想读或者写文件,你应该看open(),如果你想操作路径,你应该看os.path
模块,如果你想读取文件内的所有行,你应该看fileinput模块,如果你想操作临时文件或者目录,你应该看tempfile模块,对于较高级
别的目录和文件处理,你应该看shutil模块。

\noindent{\color{red}{os.error:}}
\par{内建OSError错误的别名。}\\

\noindent{\color{red}{os.name:}}
\par{导入依赖操作系统模块的名字。下面是目前被注册的名字:“posix”, nt”,“java”。}





\subsection{File Names, Command Line Arguments, and Environment Variables}






\subsection{Process Parameters}
\noindent{\color{red}{os.ctermid():}}
\par{返回进程控制终端的文件名。在unix中有效,请查看相关文档。}\\

\noindent{\color{red}{os.environ:}}
\par{一个mapping对象表示环境。例如,environ[“HOME”] ,表示的你自己home文件夹的路径(某些平台支持,windows不支持)
,它与C中的getenv(“HOME”)一致。}
\par{这个mapping对象在os模块第一次导入时被创建,一般在python启动时,作为site.py处理过程的一部分。在这一次之后改变environment不影响os.environ,除非直接修改os.environ。}
\par{注:putenv()不会直接改变os.environ,所以最好是修改os.environ。}
\par{注:在一些平台上,包括FreeBSD和Mac OS X,修改environ会导致内存泄露。参考 putenv()的系统文档。}
\par{如果没有提供putenv(),mapping的修改版本传递给合适的创建过程函数,将导致子过程使用一个修改的environment。}
\par{如果这个平台支持unsetenv()函数,你可以删除mapping中的项目。当从os.environ使用pop()或clear()删除一个项目时,unsetenv()会自动被调用(版本2.6)。}\\

\noindent{\color{red}{os.environb:}}
\par{environ的bytes版本。}\\

\noindent{\color{red}{os.chdir(path):}}
\par{用于改变当前的工作目录到path指定的目录。}\\

\noindent{\color{red}{os.fchdir(fd):}}
\par{用于改变当前的工作目录到fd描述的目录。}\\

\noindent{\color{red}{os.getcwd():}}
\par{返回一个用于表示当前目录的字符串。}\\

\noindent{\color{red}{os.fsencode(filename):}}
\par{未知}\\

\noindent{\color{red}{os.fsdecode(filename):}}
\par{未知}\\

\noindent{\color{red}{os.fspath(path):}}
\par{返回path所代表的文件系统。}\\

\noindent{\color{red}{os.getenv(key, default=None):}}
\par{返回environment变量key的值,返回的结果是个str。}\\

\noindent{\color{red}{os.getenvb(key, default=None):}}
\par{返回environment变量key的值,返回的结果是个bytes。}\\

\noindent{\color{red}{os.get\_exec\_path(env=None):}}
\par{未知}\\

\noindent{\color{red}{os.getegid():}}
\par{返回当前进程有效的group的id。对应于当前进程的可执行文件的“set id”的bit位。在unix中有效,请查看相关文档。}\\

\noindent{\color{red}{os.geteuid():}}
\par{返回当前进程有效的user的id。在unix中有效,请查看相关文档。}\\

\noindent{\color{red}{os.getgid():}}
\par{返回当前进程当前group的id。}\\

\noindent{\color{red}{os.getgrouplist(user, group):}}
\par{未知}\\

\noindent{\color{red}{os.getgroups():}}
\par{返回当前进程支持的groups的id列表。}\\

\noindent{\color{red}{os.getlogin():}}
\par{返回进程控制终端登陆用户的名字。在大多情况下它比使用environment变量LOGNAME来得到用户名,或使用pwd.getpwuid(os.getuid())[0] 得到当前有效用户id的登陆名更为有效。}\\

\noindent{\color{red}{os.getpgid(pid):}}
\par{返回pid进程的group id.如果pid为0,返回当前进程的group id。}\\

\noindent{\color{red}{os.getpgrp():}}
\par{返回当前进程组的id。}\\

\noindent{\color{red}{os.getpid():}}
\par{返回当前进程的id。}\\

\noindent{\color{red}{os.getppid():}}
\par{返回当前父进程的id。}\\

\noindent{\color{red}{os.getpriority(which, who):}}
\par{未知}\\

\noindent{\color{red}{os.getresuid():}}
\par{返回一个元组(ruid, euid, suid)代表当前进程的real, effective和saved user ids。}\\

\noindent{\color{red}{os.getresgid():}}
\par{返回一个元组(rgid, egid, sgid)代表当前进程的real, effective和saved group ids。}\\

\noindent{\color{red}{os.getuid():}}
\par{返回当前进程的真正用户的id。}\\

\noindent{\color{red}{os.initgroups(username, gid):}}
\par{未知。}\\






%%% Local Variables:
%%% mode: latex
%%% TeX-master: t
%%% End:
