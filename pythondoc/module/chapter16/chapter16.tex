%#-*- coding:utf-8 -*-
\chapter{Generic Operating System Services}
\section{os--Miscellaneous operating system interfaces}
该模块提供了方便的使用操作系统依赖的函数的方式。如果你想读或者写文件,你应该看open(),如果你想操作路径,你应该看os.path
模块,如果你想读取文件内的所有行,你应该看fileinput模块,如果你想操作临时文件或者目录,你应该看tempfile模块,对于较高级
别的目录和文件处理,你应该看shutil模块。

\noindent{\color{red}{os.error:}}
\par{内建OSError错误的别名。}\\

\noindent{\color{red}{os.name:}}
\par{导入依赖操作系统模块的名字。下面是目前被注册的名字:“posix”, nt”,“java”。}





\subsection{File Names, Command Line Arguments, and Environment Variables}






\subsection{Process Parameters}
\noindent{\color{red}{os.ctermid():}}
\par{返回进程控制终端的文件名。在unix中有效,请查看相关文档。}\\

\noindent{\color{red}{os.environ:}}
\par{一个mapping对象表示环境。例如,environ[“HOME”] ,表示的你自己home文件夹的路径(某些平台支持,windows不支持)
,它与C中的getenv(“HOME”)一致。}
\par{这个mapping对象在os模块第一次导入时被创建,一般在python启动时,作为site.py处理过程的一部分。在这一次之后改变environment不影响os.environ,除非直接修改os.environ。}
\par{注:putenv()不会直接改变os.environ,所以最好是修改os.environ。}
\par{注:在一些平台上,包括FreeBSD和Mac OS X,修改environ会导致内存泄露。参考 putenv()的系统文档。}
\par{如果没有提供putenv(),mapping的修改版本传递给合适的创建过程函数,将导致子过程使用一个修改的environment。}
\par{如果这个平台支持unsetenv()函数,你可以删除mapping中的项目。当从os.environ使用pop()或clear()删除一个项目时,unsetenv()会自动被调用(版本2.6)。}\\

\noindent{\color{red}{os.environb:}}
\par{environ的bytes版本。}\\

\noindent{\color{red}{os.chdir(path):}}
\par{用于改变当前的工作目录到path指定的目录。}\\

\noindent{\color{red}{os.fchdir(fd):}}
\par{用于改变当前的工作目录到fd描述的目录。}\\

\noindent{\color{red}{os.getcwd():}}
\par{返回一个用于表示当前目录的字符串。}\\

\noindent{\color{red}{os.fsencode(filename):}}
\par{未知}\\

\noindent{\color{red}{os.fsdecode(filename):}}
\par{未知}\\

\noindent{\color{red}{os.fspath(path):}}
\par{返回path所代表的文件系统。}\\

\noindent{\color{red}{os.getenv(key, default=None):}}
\par{返回environment变量key的值,返回的结果是个str。}\\

\noindent{\color{red}{os.getenvb(key, default=None):}}
\par{返回environment变量key的值,返回的结果是个bytes。}\\

\noindent{\color{red}{os.get\_exec\_path(env=None):}}
\par{未知}\\

\noindent{\color{red}{os.getegid():}}
\par{返回当前进程有效的group的id。对应于当前进程的可执行文件的“set id”的bit位。在unix中有效,请查看相关文档。}\\

\noindent{\color{red}{os.geteuid():}}
\par{返回当前进程有效的user的id。在unix中有效,请查看相关文档。}\\

\noindent{\color{red}{os.getgid():}}
\par{返回当前进程当前group的id。}\\

\noindent{\color{red}{os.getgrouplist(user, group):}}
\par{未知}\\

\noindent{\color{red}{os.getgroups():}}
\par{返回当前进程支持的groups的id列表。}\\

\noindent{\color{red}{os.getlogin():}}
\par{返回进程控制终端登陆用户的名字。在大多情况下它比使用environment变量LOGNAME来得到用户名,或使用pwd.getpwuid(os.getuid())[0] 得到当前有效用户id的登陆名更为有效。}\\

\noindent{\color{red}{os.getpgid(pid):}}
\par{返回pid进程的group id.如果pid为0,返回当前进程的group id。}\\

\noindent{\color{red}{os.getpgrp():}}
\par{返回当前进程组的id。}\\

\noindent{\color{red}{os.getpid():}}
\par{返回当前进程的id。}\\

\noindent{\color{red}{os.getppid():}}
\par{返回当前父进程的id。}\\

\noindent{\color{red}{os.getpriority(which, who):}}
\par{未知}\\

\noindent{\color{red}{os.PRIO\_PROCESS}}\\
\noindent{\color{red}{os.PRIO\_PGRP}}\\
\noindent{\color{red}{os.PRIO\_USER}}
\par{getpriority()和setpriority()的参数。}\\

\noindent{\color{red}{os.getresuid():}}
\par{返回一个元组(ruid, euid, suid)代表当前进程的real, effective和saved user ids。}\\

\noindent{\color{red}{os.getresgid():}}
\par{返回一个元组(rgid, egid, sgid)代表当前进程的real, effective和saved group ids。}\\

\noindent{\color{red}{os.getuid():}}
\par{返回当前进程的真正用户的id。}\\

\noindent{\color{red}{os.initgroups(username, gid):}}
\par{未知。}\\

\noindent{\color{red}{os.putenv(key, value):}}
\par{将环境变量名字为key的值为value。}\\

\noindent{\color{red}{os.setegid(egid):}}
\par{设置当前进程的有效组id。}\\

\noindent{\color{red}{os.seteuid(euid):}}
\par{设置当前进程的有效用户id。}\\

\noindent{\color{red}{os.setgid(gid):}}
\par{设置当前进程组的id.}\\

\noindent{\color{red}{os.setgroups(groups):}}
\par{设置当前进程支持的groupsid列表。groups必须是列表,每个元素必须是整数,这个操作只对超级用户组有效。}\\

\noindent{\color{red}{os.setpgrp():}}
\par{未知}\\

\noindent{\color{red}{os.setpgid(pid, pgrp):}}
\par{未知}\\

\noindent{\color{red}{os.setpriority(which, who, priority):}}
\par{未知}\\

\noindent{\color{red}{os.setregid(rgid, egid):}}
\par{未知}\\

\noindent{\color{red}{os.setresuid(ruid, euid, suid):}}
\par{未知}\\

\noindent{\color{red}{os.setreuid(ruid, euid):}}
\par{未知}\\

\noindent{\color{red}{os.getsid(pid):}}
\par{未知}\\

\noindent{\color{red}{os.setsid():}}
\par{未知}\\

\noindent{\color{red}{os.setuid(uid):}}
\par{未知}\\

\noindent{\color{red}{os.strerror(code):}}
\par{未知}\\

\noindent{\color{red}{os.supports\_bytes\_environ:}}
\par{未知}\\

\noindent{\color{red}{os.umask(mask):}}
\par{未知}\\

\noindent{\color{red}{os.uname():}}
\par{未知}\\

\noindent{\color{red}{os.unsetenv(key):}}
\par{未知}\\






\subsection{File Object Creation}

\noindent{\color{red}{os.fdopen(fd, *args, **kwargs):}}
\par{未知}\\





\subsection{File Descriptor Operations}

\noindent{\color{red}{os.close(fd):}}
\par{关闭文件描述符。}\\

\noindent{\color{red}{os.closerange(fd\_low, fd\_high):}}
\par{未知}\\

\noindent{\color{red}{os.device\_encoding(fd):}}
\par{未知}\\

\noindent{\color{red}{os.dup(fd):}}
\par{未知}\\

\noindent{\color{red}{os.dup2(fd, fd2, inheritable=True):}}
\par{未知}\\

\noindent{\color{red}{os.fchmod(fd, mode):}}
\par{未知}\\

\noindent{\color{red}{os.fchown(fd, uid, gid):}}
\par{未知}\\

\noindent{\color{red}{os.fdatasync(fd):}}
\par{未知}\\

\noindent{\color{red}{os.fpathconf(fd, name):}}
\par{未知}\\

\noindent{\color{red}{os.fstat(fd):}}
\par{未知}\\

\noindent{\color{red}{os.fstatvfs(fd):}}
\par{未知}\\

\noindent{\color{red}{os.fsync(fd):}}
\par{未知}\\

\noindent{\color{red}{os.ftruncate(fd, length):}}
\par{未知}\\

\noindent{\color{red}{os.get\_blocking(fd):}}
\par{未知}\\

\noindent{\color{red}{os.isatty(fd):}}
\par{未知}\\

\noindent{\color{red}{os.lockf(fd, cmd, len):}}
\par{未知}\\

\noindent{\color{red}{os.F\_LOCK:}}\\
\noindent{\color{red}{os.F\_TLOCK}}\\
\noindent{\color{red}{os.F\_ULOCK}}\\
\noindent{\color{red}{os.F\_TEST}}
\par{未知}\\

\noindent{\color{red}{os.lseek(fd, pos, how):}}
\par{未知}\\

\noindent{\color{red}{os.SEEK\_SET}}\\
\noindent{\color{red}{os.SEEK\_CUR}}\\
\noindent{\color{red}{os.SEEK\_END}}
\par{未知}\\

\noindent{\color{red}{os.open(path, flags, mode=0o777, *, dir\_fd=None):}}
\par{未知}\\

\noindent{\color{red}{os.O\_RDONLY:}}\\
\noindent{\color{red}{os.O\_WRONLY:}}\\
\noindent{\color{red}{os.O\_RDWR:}}\\
\noindent{\color{red}{os.O\_APPEND:}}\\
\noindent{\color{red}{os.O\_CREAT:}}\\
\noindent{\color{red}{os.O\_EXCL:}}\\
\noindent{\color{red}{os.O\_TRUNC:}}\\
\noindent{\color{red}{os.O\_DSYNC:}}\\
\noindent{\color{red}{os.O\_RSYNC:}}\\
\noindent{\color{red}{os.O\_SYNC:}}\\
\noindent{\color{red}{os.O\_NDELAY:}}\\
\noindent{\color{red}{os.O\_NONBLOCK:}}\\
\noindent{\color{red}{os.O\_NOCTTY:}}\\
\noindent{\color{red}{os.O\_CLOEXEC:}}\\
\noindent{\color{red}{os.O\_BINATY:}}\\
\noindent{\color{red}{os.O\_NOINHERIT:}}\\
\noindent{\color{red}{os.O\_SHORT\_LIVED:}}\\
\noindent{\color{red}{os.O\_TEMPORARY:}}\\
\noindent{\color{red}{os.O\_RANDOM:}}\\
\noindent{\color{red}{os.O\_SEQUENTIAL:}}\\
\noindent{\color{red}{os.O\_TEXT:}}\\
\noindent{\color{red}{os.O\_ASYNC:}}\\
\noindent{\color{red}{os.O\_DIRECT:}}\\
\noindent{\color{red}{os.O\_DIRECTORY:}}\\
\noindent{\color{red}{os.O\_NOFOLLOW:}}\\
\noindent{\color{red}{os.O\_NOATIME:}}\\
\noindent{\color{red}{os.O\_PATH:}}\\
\noindent{\color{red}{os.O\_TMPFILE:}}\\
\noindent{\color{red}{os.O\_SHLOCK:}}\\
\noindent{\color{red}{os.O\_EXLOCK:}}
\par{未知}\\

\noindent{\color{red}{os.openpty():}}
\par{未知}\\

\noindent{\color{red}{os.pipe():}}
\par{未知}\\

\noindent{\color{red}{os.pipe2(flags):}}
\par{未知}\\

\noindent{\color{red}{os.posix\_fallocate(fd, offset, len):}}
\par{未知}\\

\noindent{\color{red}{os.posix\_fadvise(fd, offset, len, advice):}}
\par{未知}\\

\noindent{\color{red}{os.POSIX\_FADV\_NORMAL:}}\\
\noindent{\color{red}{os.POSIX\_FADV\_SEQUENTIAL:}}\\
\noindent{\color{red}{os.POSIX\_FADV\_RANDOM:}}\\
\noindent{\color{red}{os.POSIX\_FADV\_NOREUSE:}}\\
\noindent{\color{red}{os.POSIX\_FADV\_WILLNEED:}}\\
\noindent{\color{red}{os.POSIX\_FADV\_DONTNEED:}}
\par{未知}\\

\noindent{\color{red}{os.pread(fd, buffersize, offset):}}
\par{未知}\\

\noindent{\color{red}{os.pwrite(fd, str, offset):}}
\par{未知}\\

\noindent{\color{red}{os.read(fd, n):}}
\par{未知}\\

\noindent{\color{red}{os.sendfile(out, in, offset, count):}}\\
\noindent{\color{red}{os.sendfile(out, in, offset, count, [headers,][trailers,]flags=0)}}
\par{未知}\\

\noindent{\color{red}{os.set\_blocking(fd, blocking):}}
\par{未知}\\

\noindent{\color{red}{os.SF\_NODISKIO:}}\\
\noindent{\color{red}{os.SF\_MNOWAIT:}}\\
\noindent{\color{red}{os.SF\_SYNC:}}
\par{未知}\\

\noindent{\color{red}{os.read(fd, buffers):}}
\par{未知}\\

\noindent{\color{red}{os.tcgetpgrp(fd):}}
\par{未知}\\

\noindent{\color{red}{os.tcsetpgrp(fd, pg):}}
\par{未知}\\

\noindent{\color{red}{os.ttyname(fd):}}
\par{未知}\\

\noindent{\color{red}{os.write(fd, str):}}
\par{未知}\\

\noindent{\color{red}{os.writev(fd, buffers):}}
\par{未知}\\

\noindent{\color{red}{os.get\_terminal\_size(fd=STDOUT\_FILENO):}}
\par{未知}\\

\noindent{\color{red}{os.terminal\_size:}}
\par{未知}\\

\noindent{\color{red}{os.get\_inheritable(fd):}}
\par{未知}\\

\noindent{\color{red}{os.set\_inheritable(fd, inheritable):}}
\par{未知}\\

\noindent{\color{red}{os.get\_handle\_inheritable(handle):}}
\par{未知}\\

\noindent{\color{red}{os.set\_handle\_inheritable(handle, inheritable):}}
\par{未知}\\

\noindent{\color{red}{os.access(path, mode, *, dir\_fd=None, effective\_ids=False,\\ follow\_symlinks=True):}}
\par{未知}\\

\noindent{\color{red}{os.F\_OK:}}
\noindent{\color{red}{os.R\_OK:}}
\noindent{\color{red}{os.W\_OK:}}
\noindent{\color{red}{os.X\_OK:}}
\par{未知}\\

\noindent{\color{red}{os.chdir(path):}}
\par{未知}\\

\noindent{\color{red}{os.chflags(path, flags, *, follows\_symlinks=True):}}
\par{未知}\\

\noindent{\color{red}{os.chmod(path, mode, *, dir\_fd=None, follow\_symlinks=True):}}
\par{未知}\\

\noindent{\color{red}{os.chown(path, uid, gid, *, dir\_fd=None, follow\_symlinks=True):}}
\par{未知}\\

\noindent{\color{red}{os.chroot(path):}}
\par{未知}\\

\noindent{\color{red}{os.fchdir(fd):}}
\par{未知}\\

\noindent{\color{red}{os.getcwd():}}
\par{未知}\\

\noindent{\color{red}{os.getcwdb():}}
\par{未知}\\

\noindent{\color{red}{os.lchflags(path, flags):}}
\par{未知}\\

\noindent{\color{red}{os.lchmod(path, mode):}}
\par{未知}\\

\noindent{\color{red}{os.lchown(path, uid, gid):}}
\par{未知}\\

\noindent{\color{red}{os.link(src, dst, *, src\_dir\_fd=None, dst\_dir\_fd=None,\\ follow\_symlinks=True):}}
\par{未知}\\

\noindent{\color{red}{os.listdir(path=\textquoteleft{.}\textquoteright{}):}}
\par{未知}\\

\noindent{\color{red}{os.lstat(path, *, dir\_fd=None):}}
\par{未知}\\

\noindent{\color{red}{os.mkdir(path, mode=0o777, *, dir\_fd=None):}}
\par{未知}\\

\noindent{\color{red}{os.makedirs(name, mode=0o777, exist\_ok=False):}}
\par{未知}\\

\noindent{\color{red}{os.mkfifo(path, mode=0o666, *, dir\_fd=None):}}
\par{未知}\\

\noindent{\color{red}{os.mknod(path, mode=0o666, device=0, *, dir\_fd=None):}}
\par{未知}\\

\noindent{\color{red}{os.major(device):}}
\par{未知}\\

\noindent{\color{red}{os.minor(device):}}
\par{未知}\\

\noindent{\color{red}{os.makedev(major, minor):}}
\par{未知}\\

\noindent{\color{red}{os.pathconf(path, name):}}
\par{未知}\\

\noindent{\color{red}{os.pathconf\_names:}}
\par{未知}\\

\noindent{\color{red}{os.readlink(path, *, dir\_fd=None):}}
\par{未知}\\

\noindent{\color{red}{os.remove(path, *, dir\_fd=None):}}
\par{未知}\\

\noindent{\color{red}{os.removedirs(name):}}
\par{未知}\\

\noindent{\color{red}{os.rename(src, dst, *, src\_dir\_fd=None, dst\_dir\_fd=None):}}
\par{未知}\\

\noindent{\color{red}{os.renames(old, new):}}
\par{未知}\\

\noindent{\color{red}{os.replace(src, dst, *, src\_dir\_fd=None, dst\_dir\_fd=None):}}
\par{未知}\\

\noindent{\color{red}{os.rmdir(path, *, dir\_fd=None):}}
\par{未知}\\

\noindent{\color{red}{os.scandir(path=\textquoteleft{.}\textquoteright{}):}}
\par{未知}\\

\noindent{\color{red}{os.stat(path, *, dir\_fd=None, follow\_symlinks=True):}}
\par{Get the status of a file descriptor. Perform the quivalent of a stat() system call on the
given path. path may be specified as either a string or bytes ----directly or indirectly
through the Pathlike interface ---- or as an open file descriptor. Return a stat\_result
object.}\\
\begin{lstlisting}[language=Python]
>>> import os
>>> statinfo = os.stat('somefile.txt')
>>> statinfo
os.stat_result(st_mode = 33188, st_ino = 7876932, st_dev = 23456,
st_mlink = 1, st_uid = 501, st_gid = 501, st_size = 264, st_atime=123,
st_mtime=129, st_ctime=123)
>>> statinfo.st_size
264
\end{lstlisting}
\begin{itemize}
\item[*] st\_mode : file mode: file type and file mode bits.
\item[*] st\_ino : Inode number.
\item[*] st\_dev : Identifier of the device on which this file resides.
\item[*]{st\_nlink : Number of hard links.}
\item[*]{st\_uid : User identifier of the file owner.}
\item[*]{st\_gid : Group identifier of the file owner.}
\item[*]{st\_size : Size of the file in bytes.}
\item[*]{st\_atime : Time of most recent access expressed in seconds.}
\item[*]{st\_mtime : Time of most recent content modification expressed in seconds.}
\item[*]{st\_ctime : The time of most recent metadata change on Unix.}
\item[*]{st\_atime\_ns : Time of most recent access expressed in nanoseconds as an integer.}
\item[*]{st\_mtime\_ns : Time of most recent modification expressed in nanoseconds as an integer.}
\item[*]{st\_ctime\_ns : Time of most recent metadata change on Unix.}
\item[*]{st\_blocks : Number of 512-byte blocks allocated for file.}
\item[*]{st\_blksize : \textquotedblleft{Preferred}\textquotedblright{} blocksize for efficient
file system I/O. Writing to a file in smaller chunks may cause an inefficient read-modify-rewrite.}
\item[*]{st\_rdev : Type of device if an inode device.}
\item[*]{st\_flags : User defined flags for file.}
\end{itemize}


\noindent{\color{red}{os.stat\_float\_times([newvalue]):}}
\par{未知}\\

\noindent{\color{red}{os.statvfs(path):}}
\par{未知}\\

\noindent{\color{red}{os.supports\_dir\_fd:}}
\par{未知}\\

\noindent{\color{red}{os.supports\_effective\_ids:}}
\par{未知}\\

\noindent{\color{red}{os.supports\_fd:}}
\par{未知}\\

\noindent{\color{red}{os.supports\_follow\_symlinks:}}
\par{未知}\\

\noindent{\color{red}{os.symlinks(src, dst, target\_is\_directory=False, *, dir\_fd=None):}}
\par{未知}\\

\noindent{\color{red}{os.sync():}}
\par{未知}\\

\noindent{\color{red}{os.truncate(path, length):}}
\par{未知}\\

\noindent{\color{red}{os.unlink(path, *, dir\_fd=None):}}
\par{未知}\\

\noindent{\color{red}{os.utime(path, times=None, *, [ns, ]dir\_fd=None, follow\_symlinks=True):}}
\par{未知}\\

\noindent{\color{red}{os.walk(top, topdown=True, onerror=None, followlinks=False):}}
\par{未知}\\

\noindent{\color{red}{os.fwalk(top=\textquoteleft{.}\textquoteright{}, topdown=True, onerror=None, *, follow\_symlinks=False, dir\_fd=None):}}
\par{未知}\\

\noindent{\color{red}{os.getattr(path, attribute, *, follow\_symlinks=True):}}
\par{未知}\\

\noindent{\color{red}{os.listxattr(path=None, *, follow\_symlinks=True):}}
\par{未知}\\

\noindent{\color{red}{os.listattr(path=None, *, follow\_symlinks=True):}}
\par{未知}\\

\noindent{\color{red}{os.removexattr(path, attribute, *, follow\_symlinks=True):}}
\par{未知}\\

\noindent{\color{red}{os.setxattr(path, attribute, value, flags=0, *, follow\_symlinks=True):}}
\par{未知}\\

\noindent{\color{red}{os.XATTR\_SIZE\_MAX:}}
\par{未知}\\

\noindent{\color{red}{os.XATTR\_SIZE\_CREATE:}}
\par{未知}\\

\noindent{\color{red}{os.XATTR\_SIZE\_REPLACE:}}
\par{未知}\\

\noindent{\color{red}{os.abort():}}
\par{未知}\\

\noindent{\color{red}{os.execl(path, arg0, arg1, \dots):}}
\par{未知}\\

\noindent{\color{red}{os.execle(path, arg0, arg1, \dots, env):}}
\par{未知}\\

\noindent{\color{red}{os.execlp(file, arg0, arg1, \dots):}}
\par{未知}\\

\noindent{\color{red}{os.execlpe(file, arg0, arg1, \dots, env):}}
\par{未知}\\

\noindent{\color{red}{os.execv(path, args):}}
\par{未知}\\

\noindent{\color{red}{os.execve(path, args, env):}}
\par{未知}\\

\noindent{\color{red}{os.execvp(file, args):}}
\par{未知}\\

\noindent{\color{red}{os.execvpe(file, args, env):}}
\par{未知}\\

\noindent{\color{red}{os.\_exit(n):}}
\par{未知}\\
\begin{itemize}
\item[*]{os.EX\_OK : Exit code that means no error occurred.}
\item[*]{os.EX\_USAGE : Exit code that means the command was used incorrectly, such as when the
wrong number of arguments are given.}
\item[*]{os.EX\_DATAERR : Exit code that means the input data was incorrect.}
\item[*]{os.EX\_NOTINPUT : Exit code that means an input file did not exist or was not readable.}
\item[*]{os.EX\_NOUSER : Exit code that means a specified user did not exist.}
\item[*]{os.EX\_NOHOST : Exit code that means a specified host did not exist.}
\item[*]{os.EX\_UNAVAILABLE : Exit code that means that a required service is unavailable.}
\item[*]{os.EX\_SOFTWARE : Exit code that means an internal software error was detected.}
\item[*]{os.EX\_OSERR : Exit code that means an operating system error was detected, such as the
inability to fork or create a pipe.}
\item[*]{os.EX\_OSFILE : Exit code that means some system file did not exist, could not be opened,
or had some other kind of error.}
\item[*]{os.EX\_CANTCREATE : Exit code that means a user specified output file could not be
created.}
\item[*]{os.EX\_IOERR : Exit code that means that an error occurred while doing I/O on some file.}
\item[*]{os.EX\_TEMPFAIL : Exit code that means a temporary failure occurred.}
\item[*]{os.EX\_PROTOCOL : Exit code that means that a protocol exchange was illegal, invalid, or
not understood.}
\item[*]{os.EX\_NOPERM : Exit code that means that there were insufficient permissions to perform
the operation (but not intended for file system problems).}
\item[*]{os.EX\_CONFIG : Exit code that means that some kind of configuration error occurred.}
\item[*]{os.EX\_NOTFOUND : Exit code that means something like \textquotedblleft{an entry was not found}\textquotedblright{}.}
\end{itemize}

\noindent{\color{red}{os.folk():}}
\par{创建子进程。}\\

\noindent{\color{red}{os.forkpty():}}
\par{未知}\\

\noindent{\color{red}{os.kill(pid, sig):}}
\par{对进程pid发送信号值sig.}\\

\noindent{\color{red}{os.killpg(pgid, sig):}}
\par{未知}\\

\noindent{\color{red}{os.nice(increment):}}
\par{未知}\\

\noindent{\color{red}{os.plock(op):}}
\par{未知}\\

\noindent{\color{red}{os.popen(cmd, mode=\textquoteleft{r}\textquoteright{}, buffering=-1):}}
\par{未知}\\

\noindent{\color{red}{os.spawnl(mode, path, \dots):}}\\
\noindent{\color{red}{os.spawnle(mode, path, \dots, env):}}\\
\noindent{\color{red}{os.spawnlp(mode, file, \dots):}}\\
\noindent{\color{red}{os.spawnlpe(mode, file, \dots, env):}}\\
\noindent{\color{red}{os.spawnv(mode, path, args):}}\\
\noindent{\color{red}{os.spawnve(mode, path, args, env):}}\\
\noindent{\color{red}{os.spawnvp(mode, file, args)}}\\
\noindent{\color{red}{os.spawnvpe(mode, file, args, env)}}
\par{未知}\\



%%% Local Variables:
%%% mode: latex
%%% TeX-master: t
%%% End:
