%#-*- coding:utf-8 -*-
\chapter{内建函数}

\noindent{\color{red}{abs(x):}}
\par{返回x的绝对值。}\\

\noindent{\color{red}{all(iterable):}}
\par{当可迭代变量iterable的所有元素为真时返回True。}\\

\noindent{\color{red}{any(iterable):}}
\par{当可迭代变量iterable的任何一个元素为真时返回True。}\\

\noindent{\color{red}{ascii(object):}}
\par{返回一个可打印的对象的字符串表示,当遇到非ASCII码时,输出\textbackslash{u},\textbackslash{x}或\textbackslash{U}等字符。}\\

\noindent{\color{red}{bin(x):}}
\par{将一个int的整数转为二进制字符串。}\\

\noindent{\color{red}{class bool([x]):}}
\par{返回一个bool值。}\\

\noindent{\color{red}{bytearray([source[, encoding[, errors]]]):}}
\par{未知}\\

\noindent{\color{red}{bytes([source[, encoding[, errors]]]):}}
\par{未知}\\

\noindent{\color{red}{callable(object):}}
\par{检查object是否是可调用的,返回True或False。}\\

\noindent{\color{red}{chr(i):}}
\par{返回整数i对应的ASCII字符。}\\

\noindent{\color{red}{classmethod(function):}}
\par{未知}\\

\noindent{\color{red}{compile(source, filename, mode, flags=0, dont\_inherit=False, optimize=$-1$):}}
\par{将source编译为代码或AST对象,代码对象可以被exec()或eval()执行。}\\

\noindent{\color{red}{complex([real[,imag]]):}}
\par{生成一个复数对象}\\

\noindent{\color{red}{dir([object]):}}
\par{无参数时,返回当前局部域的变量名列表,有参数时,则尝试返回该对象的有效属性列表。}\\

\noindent{\color{red}{divmod(a, b):}}
\par{取两个数字(非复数)作为参数,并返回一对数,其中包含他们的商和余数。}\\

\noindent{\color{red}{enumerate(sequence, [start=0]):}}
\par{将可循环序列sequence以start开始分别列出序列的数据下标和序列数据,即对一个可遍历的数据对象(如
列表、元组或字符串),enumerate会将该数据对象组合为一个索引序列,同时列出数据的下标和数据的内容。}\\






%%% Local Variables:
%%% mode: latex
%%% TeX-master: t
%%% End:
