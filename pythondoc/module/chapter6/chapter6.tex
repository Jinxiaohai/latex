%#-*- coding:utf-8 -*-
\chapter{文本处理服务}
\section{string}




\section{re}
该模块提供了正则表达式的处理操作。
Both patterns and strings to be searched can be Unicode strings as well as 8-bit strings. However, Unicode strings and 8-bit strings cannot be mixed: that is, you cannot match a Unicode string with a byte pattern or vice-versa; similarly,when asking for a substitution, the replacement string must be of the same type as both the pattern and the search string.
\par
Regular expressions use the backslash character \textquoteleft{}\textbackslash{}\textquoteright{}) to indicate special forms or to allow special characters to be used without invoking their special meaning. This collides with Python’s usage of the same character for the same purpose in string literals; for example, to match a literal backslash, one might have to write \textquoteleft{}\textbackslash{}\textbackslash{}\textbackslash{}\textbackslash{}\textquoteright{} as the pattern string, because the regular expression must be \textbackslash{}\textbackslash{}, and each backslash must be expressed as \textbackslash{}\textbackslash{} inside a regular Python string literal.
\par
The solution is to use Python’s raw string notation for regular expression patterns; backslashes are not handled in any
special way in a string literal prefixed with \textquoteleft{r}\textquoteright{}. So r\textquotedblleft{}\textbackslash{n}\textquotedblright{} is a two-character string containing \textquoteleft{}\textbackslash{}\textquoteright{} and \textquoteleft{n}\textquoteright{}, while
\textquotedblleft{}\textbackslash{n}\textquoteright{} is a one-character string containing a newline. Usually patterns will be expressed in Python code using this raw
string notation.\\
\par
正则表达式的基本语法:
\begin{enumerate}
\item \textquoteleft{.}\textquoteright{}:在缺省的模式下,该符号匹配除了换行符外(\textquoteleft{}\textbackslash{n}\textquoteright{})的任何字符。
\item \textquoteleft{}\^{}\textquoteright{}:匹配字符串的起始部分。
\item \textquoteleft{}\$\textquoteright{}:匹配字符串的终止部分。
\item \textquoteleft{}*\textquoteright{}:匹配0次或者多次前面出现的正则表达式。
\item \textquoteleft{}+\textquoteright{}:匹配1次或者多次前面出现的正则表达式。
\item \textquoteleft{}?\textquoteright{}:匹配0次或者1次前面出现的正则表达式。
\item \textquoteleft{}\{m\}\textquoteright{}:匹配m次前面出现的正则表达式。
\item \textquoteleft{}\{m, n\}\textquoteright{}:匹配m$\sim$次前面出现的正则表达式。
\item $\left[\dots\right]$:匹配来自字符集的任意单一字符。
\item \textquoteleft{}$|$\textquoteright{}:A$|$B匹配A,B之间的任意一个。
\item (\dots):匹配封闭的正则表达式,然后另存为子组。
\item (?\dots):未知。
\item (?aiLmsux):在正则表达式中嵌入一个或者多个特殊\textquotedblleft{标记}\textquotedblright{}参数。
\item (?:\dots):表示一个匹配但不用保存的分组。
\item (?imsx-imsx:\dots):未知。
\item (?P\texttt{<}name\texttt{>}\dots{}):像一个仅由name标记而不是数字ID标识的正则分组匹配。
\end{enumerate}

\subsection{Module methods}
该模块定义了几个函数,常量和一个异常类型。\\

\noindent{\color{red}{re.compile(pattern, flags=0):}}
\par{编译一个正则表达式模式到一个正则表达式对象中,返回的就是该正则表达式对象,该对象拥有以下方法 : match() 和 search().}
\begin{lstlisting}[language=Python]
  prog = re.compile(pattern)
  result = prog.match(string)
等价于
  result = re.match(pattern, string)
\end{lstlisting}

\noindent{\color{red}{re.A:}}\\
\noindent{\color{red}{re.ASCII:}}
\par{使$\backslash$w, $\backslash$W, $\backslash$b, $\backslash$B, $\backslash$d, $\backslash$D
  $\backslash$s和$\backslash$S只采用ASCII匹配,而不是Unicode匹配。该方法只对Unicode
patterns有意义。}\\

\noindent{\color{red}{re.DEBUG:}}
\par{展示关于compiled表达式的debug信息}\\

\noindent{\color{red}{re.I:}}\\
\noindent{\color{red}{re.IGNORECASE:}}
\par{完成对字符大小写不敏感的匹配。}\\

\noindent{\color{red}{re.L:}}\\
\noindent{\color{red}{re.LOCAL:}}
\par{使$\backslash$w, $\backslash$W, $\backslash$b, $\backslash$B,
$\backslash$s和$\backslash$S以来当亲的locale.}\\

\noindent{\color{red}{re.M:}}\\
\noindent{\color{red}{re.MULTILINE:}}
\par{未知}\\

\noindent{\color{red}{re.S:}}\\
\noindent{\color{red}{re.DOTALL:}}
\par{未知}\\

\noindent{\color{red}{re.X:}}
\noindent{\color{red}{re.VERBOSE:}}
\par{未知}\\

\noindent{\color{red}{re.seach(pattern, string, flags=0):}}
\par{search()会用它的字符串参数,在任意位置对给定的正则表达式模式搜索第一次出现的匹配情况,如果
搜索到匹配成功的匹配,就会返回一个匹配对象,否则,返回None,调用返回对象的group()函数得到匹配的
结果的str表示。}\\

\noindent{\color{red}{re.match(pattern, string, flags=0):}}
\par{match()函数试图从字符串的起始部分对模式进行匹配。如果匹配成功,就返回一个匹配对象;如果匹
配失败,就返回None,匹配对象的group()方法能够用来显示那个成功的匹配。}\\
\begin{itemize}
\item[*]{group() : 匹配对象的group()方法能够用来显示那个成功的匹配。}
\item[*]{start() : 返回匹配开始的位置。}
\item[*]{end() : 返回匹配结束的位置。}
\item[*]{span() : 返回一个元组包含匹配(开始,结束)的位置。}
\end{itemize}

\noindent{\color{red}{re.fullmatch(pattern, string, flags=0):}}
\par{必须完全匹配才返回值,可以指定字符串的一段位置进行来匹配。}
\begin{lstlisting}[language=Python]
  patter = re.compile("o[gh]")
  print(pattern,fullmatch("dog")) #返回None,没有og|oh开头。
  print(pattern.fullmatch("ohr")) #返回None,不是整串完全匹配,虽然有oh开头,但是还包含字母r。
  print(pattern.fullmatch("og"))  #返回og,完全匹配
  print(pattern.fullmatch("doggie", 1, 3)) #返回og,完全匹配。
\end{lstlisting}

\noindent{\color{red}{re.split(pattern, string, maxsplit=0, flags=0):}}
\par{字符串分割,str.split只能按照某个分割符分割,正则表达式可以按照某个规则进行分割。}\\

\noindent{\color{red}{re.findall(pattern, string, flags=0):}}
\par{获取全部的匹配字符,返回一个所有匹配字符串的列表。}\\

\noindent{\color{red}{re.finditer(pattern, string, flags=0):}}
\par{与findall类似,只是返回的是一个迭代器。}\\

\noindent{\color{red}{re.sub(pattern, repl, string, count=0, flags=0):}}
\par{sub()将根据正则表达式找到的字符串用新的字符串进行替换,返回结果为字符串。}
\begin{itemize}
\item[*]{pattern : 正则表达式。}
\item[*]{repl : 替换的字符串。}
\item[*]{string : 源字符串。}
\item[*]{count : 表示要替换的个数。}
\end{itemize}

\noindent{\color{red}{re.subn(pattern, repl, string, count=0, flags=0):}}
\par{和sub()的作用相同,单数返回的是一个元组(new\_string, number\_of\_subs\_made)}\\

\noindent{\color{red}{re.escape(string):}}
\par{对字符串中的非字母数字进行转义。}\\

\noindent{\color{red}{re.purge():}}
\par{清空缓存中的正则表达式。}\\




\subsection{返回的匹配对象的操作}

Match objects support the following methods and attributes.
\noindent{\color{red}{match.expand(template):}}
\par{Return the string obtained by doing backslash substitution on the template string template.}\\

\noindent{\color{red}{match.group([group1, \dots]):}}
\par{无参数的时候返回整个的字符串,注意有参数时候的结果的意义。}\\

\noindent{\color{red}{match.groups(default=None):}}
\par{这个自己慢慢的看下,要认真。}\\

\noindent{\color{red}{match.groupdict(default=None):}}
\par{未知}\\

\noindent{\color{red}{match.start([group]):}}
\par{未知}\\

\noindent{\color{red}{match.end([group]):}}
\par{未知}\\

\noindent{\color{red}{match.span([group]):}}
\par{未知}\\

\noindent{\color{red}{match.pos:}}
\par{未知}\\

\noindent{\color{red}{match.endpos:}}
\par{未知}\\

\noindent{\color{red}{match.lastindex:}}
\par{未知}\\

\noindent{\color{red}{match.lastgroup:}}
\par{未知}\\

\noindent{\color{red}{match.re:}}
\par{未知}\\

\noindent{\color{red}{match.string:}}
\par{未知}\\


\section{difflib}





\section{textwrap}





\section{unicodedata}





\section{stringprep}





\section{readline}





\section{rlcompleter}






%%% Local Variables:
%%% mode: latex
%%% TeX-master: t
%%% End:
