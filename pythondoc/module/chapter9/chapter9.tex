%#-*- coding:utf-8 -*-
\chapter{数值和数学}
\section{numbers}




\section{math--Mathematical functions}
\subsection{Number-theoretic and representation functions}
\noindent{\color{red}{math.ceil(x):}}
\par{返回大于等于x的最小整数。}\\

\noindent{\color{red}{math.copysign(x, y):}}
\par{返回与y同号的x的值。}\\

\noindent{\color{red}{math.fabs(x):}}
\par{返回x的绝对值。}\\

\noindent{\color{red}{math.factorial(x):}}
\par{返回x的阶乘。}\\

\noindent{\color{red}{math.floor(x):}}
\par{返回小于等于x的最大整数。}\\

\noindent{\color{red}{math.fmod(x, y):}}
\par{返回x对y取模的余数,fmod类似\%,但是产生的结果可能与\%不同,因为前者以y来决定余数的符号,后者以x来决定余数的符号。}\\

\noindent{\color{red}{math.frexp(x):}}
\par{返回一个2元组分别是底数以及一个指数,也就是$x=m*2^n$。}\\

\noindent{\color{red}{math.fsun(iterable):}}
\par{返回x序列值的和。}\\

\noindent{\color{red}{math.gcd(a, b):}}
\par{返回a与b的最大公约数。}\\

\noindent{\color{red}{math.isclose(a, b, *, rel\_tol=1e-09, abs\_tol=0.0):}}
\par{如果a和b是如此的接近(小于rel\_tol的值),则返回True,否则返回Flase。}\\

\noindent{\color{red}{math.isfinite(x):}}
\par{如果x是数字,且是有限的,则返回True,否则返回Flase。}\\

\noindent{\color{red}{math.isinf(x):}}
\par{如果x是正无穷或者负无穷,则返回True,否则返回Flase。}\\

\noindent{\color{red}{math.isnan(x):}}
\par{如果x不是数字,则返回True,否则返回Flase。}\\

\noindent{\color{red}{math.ldexp(x, i):}}
\par{返回x*($2^i$)。}\\

\noindent{\color{red}{math.modf(x):}}
\par{返回x的分数部分和整数部分。}\\

\noindent{\color{red}{math.trunc(x):}}
\par{返回x的整数部分,等同于int()。}\\

\subsection{Power and logarithmic functions}
\noindent{\color{red}{math.exp(x):}}
\par{返回$e^x$。}\\

\noindent{\color{red}{math.expm1(x):}}
\par{返回$e^x-1$。}\\

\noindent{\color{red}{math.log(x[, base]):}}
\par{返回以base为底数的对数。}\\

\noindent{\color{red}{math.loglp(x):}}
\par{返回$log_{e}(1+x)$。}\\

\noindent{\color{red}{math.log2(x):}}
\par{返回math.log(x, 2)。}\\

\noindent{\color{red}{math.log10(x):}}
\par{返回math.log(x, 10)。}\\

\noindent{\color{red}{math.pow(x, y):}}
\par{返回$x^y$。}\\

\noindent{\color{red}{math.sqrt(x):}}
\par{返回x的均方根。}\\

\subsection{三角函数}
\noindent{\color{red}{math.acos(x):}}
\par{返回x的反余弦。}\\

\noindent{\color{red}{math.asin(x):}}
\par{返回x的反正弦。}\\

\noindent{\color{red}{math.atan(x):}}
\par{返回x的反正切。}\\

\noindent{\color{red}{math.atan2(y, x):}}
\par{返回y/x的反正切。}\\

\noindent{\color{red}{math.cos(x):}}
\par{返回x的余弦。}\\

\noindent{\color{red}{math.hypot(x, y):}}
\par{返回(x, y)距离原点的欧几里得距离。}\\

\noindent{\color{red}{math.sin(x):}}
\par{返回x的正弦。}\\

\noindent{\color{red}{math.tan(x):}}
\par{返回的正切。}\\

\subsection{角度变换}
\noindent{\color{red}{math.degrees(x):}}
\par{将x由弧度转换为角度。}\\

\noindent{\color{red}{math.radians(x):}}
\par{将x由角度转换为弧度。}\\

\subsection{双曲函数}
\noindent{\color{red}{math.acosh(x):}}
\par{返回x的反双曲余弦。}\\

\noindent{\color{red}{math.asinh(x):}}
\par{返回x的反双曲正弦。}\\

\noindent{\color{red}{math.atanh(x):}}
\par{返回x的反双曲正切。}\\

\noindent{\color{red}{math.cosh(x):}}
\par{返回x的双曲余弦。}\\

\noindent{\color{red}{math.sinh(x):}}
\par{返回x的双曲正弦。}\\

\noindent{\color{red}{math.tanh(x):}}
\par{返回x的双曲正切。}\\

\subsection{特殊函数}
\noindent{\color{red}{math.erf(x):}}
\par{未知}\\

\noindent{\color{red}{math.erfc(x):}}
\par{未知}\\

\noindent{\color{red}{math.gamma(x):}}
\par{返回x的gamma函数。}\\

\noindent{\color{red}{math.lgamma(x):}}
\par{未知}\\

\subsection{常数}
\noindent{\color{red}{math.pi:}}
\par{常数$\pi$。}\\

\noindent{\color{red}{math.e:}}
\par{自然常数。}\\

\noindent{\color{red}{math.inf:}}
\par{正无穷的浮点数。}\\

\noindent{\color{red}{math.nan:}}
\par{浮点型的非数字。}\\





\section{cmath-Mathematical functions for complex numbers}
\subsection{Conversions to and from polar coordinates}
\noindent{\color{red}{cmath.phase(x):}}
\par{返回x的相位角,等同于math.atan2(x.img, x.real)。}\\

\noindent{\color{red}{cmath.polar(x):}}
\par{返回在极坐标系下x的值,返回的是一个对(r, phi),其中r是x的模,phi是方位角。x.polar(x)等同于(abs(x), phase(x))。}\\

\noindent{\color{red}{cmath.rect(r, phi):}}
\par{返回一个复数,其数值为r*(math.cos(phi) + math.sin(phi)*1j)。}\\

\subsection{Power and logarithmic functions}
\noindent{\color{red}{cmath.exp(x):}}
\par{返回$e^x$。}\\

\noindent{\color{red}{cmath.log(x[, base]):}}
\par{对于给定的底数base,返回x的对数。}\\

\noindent{\color{red}{cmath.log10(x):}}
\par{返回cmath.log(x, 10)。}\\

\noindent{\color{red}{cmath.sqrt(x):}}
\par{返回x的均方根。}\\

\subsection{三角函数}
\noindent{\color{red}{cmath.acos(x):}}
\par{返回x的反余弦。}\\

\noindent{\color{red}{cmath.asin(x):}}
\par{返回x的反正弦。}\\

\noindent{\color{red}{cmath.atan(x):}}
\par{返回x的反正切。}\\

\noindent{\color{red}{cmath.cos(x):}}
\par{返回x的余弦。}\\

\noindent{\color{red}{cmath.sin(x):}}
\par{返回x的正弦。}\\

\noindent{\color{red}{cmath.tan(x):}}
\par{返回x的正切。}\\

\subsection{双曲函数}
\noindent{\color{red}{cmath.acosh(x):}}
\par{返回x的反双曲余弦。}\\

\noindent{\color{red}{cmath.asinh(x):}}
\par{返回x的反双曲正弦。}\\

\noindent{\color{red}{cmath.atanh(x):}}
\par{返回x的反双曲正切。}\\

\noindent{\color{red}{cmath.cosh(x):}}
\par{返回x的双曲余弦。}\\

\noindent{\color{red}{cmath.sinh(x):}}
\par{返回x的双曲正弦。}\\

\noindent{\color{red}{cmath.tanh(x):}}
\par{返回x的双曲正切。}\\

\subsection{Classification functions}
同math模块。
\subsection{常数}
同math模块。



\section{decimal-Decimal fixed point and floating point arithmetic}





\section{fractions--Rational numbers}





\section{random--Generate pseudo-random numbers}
\subsection{Bookkeeping functions}
\noindent{\color{red}{random.seed(a=None, version=2):}}
\par{设置随机数的起始种子,如果a省略,将会使用系统当前的时间作为种子。}\\

\noindent{\color{red}{random.getstate():}}
\par{未知}\\

\noindent{\color{red}{random.setstate(state):}}
\par{未知}\\

\noindent{\color{red}{random.getrandbits(k):}}
\par{未知}\\

\subsection{Functions for integers}
\noindent{\color{red}{random.randrange(stop):}}
\par{返回指定递增基数集合中的一个随机数,基数的缺省值为1。}\\

\noindent{\color{red}{random.randrange(start, stop[, step]):}}
\par{返回指定递增基数集合中的一个随机数,start指定范围内的起始值,end指定范围的结束值(左闭右开),step指定递增基数。}\\

\noindent{\color{red}{random.randint(a, b):}}
\par{返回[a, b]之间的一个随机整数,等同于random.randrange(a, b+1)。}\\

\subsection{Functions for sequences}
\noindent{\color{red}{random.choice(seq):}}
\par{从序列的元素中随机挑选一个元素。}\\

\noindent{\color{red}{random.choices(population, weights=None, *, cum\_weights=None, k=1):}}
\par{未知}\\

\noindent{\color{red}{random.shuffle(x[, random]):}}
\par{将序列的所有元素随机排序。}\\

\noindent{\color{red}{random.sample(population, k):}}
\par{从序列中随机挑选k个元素。}\\

\subsection{Real-valued distributions}
\noindent{\color{red}{random.random():}}
\par{随机生成下一个实数,它在[0, 1)范围内。}\\

\noindent{\color{red}{random.uniform(a, b):}}
\par{返回一个随机数,它在[a, b]分为内。}\\

\noindent{\color{red}{random.triangular(low, high, mode):}}
\par{未知}\\

\noindent{\color{red}{random.betavariate(alpha, beta):}}
\par{未知}\\

\noindent{\color{red}{random.expovariate(lambda):}}
\par{未知}\\

\noindent{\color{red}{random.gammavariate(alpha, beta):}}
\par{未知}\\

\noindent{\color{red}{random.gauss(mu, sigma):}}
\par{未知}\\

\noindent{\color{red}{random.lognormvariate(mu, sigma):}}
\par{未知}\\

\noindent{\color{red}{random.normalvariate(mu, sigma):}}
\par{未知}\\

\noindent{\color{red}{random.vonmisesvariate(mu, kappa):}}
\par{未知}\\

\noindent{\color{red}{random.paretovariate(alpha):}}
\par{未知}\\

\noindent{\color{red}{random.weibullvariate(alpha, beta):}}
\par{未知}\\








\section{statistics--Mathematical statistics functions}
%%% Local Variables:
%%% mode: latex
%%% TeX-master: t
%%% End:
