%#-*- coding:utf-8 -*-
\chapter{Internet Protocols and Support}
\section{webbrower}






\section{cgi}






\section{cgitb}






\section{wsgiref}






\section{urllib}






\section{urllib.request}
The urllib.request module defines functions and classes which help in opening URLs (mostly HTTP) in
a complex world ----basic and digest authentication, redirections, cookies and more.\\
The urllib.request模块包含以下函数:\\

\noindent{\color{red}{urllib.request.urlopen(url, data=None, [timeout,]*, cafile=None, capath=None, cadefault=False, context=None)\\
:}}
\par{打开url,其中url可以是字符串也可以是Request对象。}\\
\begin{itemize}
\item[.]{data :是一个定义了需要发给server端数据的对象或者None,一个标准的
application/x-www-from-urlencoded格式的buffer.urllib.request模块使用HTTP/1.1协议,并且在HTTP请求的报
头中包含Connection:close.}\\
\item[.]{timeout :设定连接的超时时间,如果没有设定的话就使用全局超时时间,仅对HPPT,HPPTS,
FTP连接有效。}
\item[.]{context: 如果定义了,则必须是一个描述SSL选项的ss.SSLContext实例,查看HPPTSConnection
相关资料获取详细描述。}
\item[.]{cafile和capath:参数定义了HTTPS请求中的可信任CA证书,cadefault参数可以省略。}
\end{itemize}

调用该方法返回一个像context manager一样工作且包含如下方法的对象(http.client.HTTPSResponse的实例):\\
\begin{itemize}
\item[.]{gerurl() : 返回一个资源索引URL,决定是否需要重定向。}
\item[.]{info() : 返回有关HTTP报头的元信息。}
\item[.]{getcode() : 获取响应码通。}
\end{itemize}

\noindent{\color{red}{urllib.request.install\_opener(opener):}}
\par{安装一个openerDirector实例作为默认的opener.}\\

\noindent{\color{red}{urllib.request.build\_opener([handler, \dots]):}}
\par{返回一个OpenerDirector的实例。}\\

\noindent{\color{red}{urllib.request.pathname2url(path):}}
\par{转换路径到URL,默认为/,并不会生成一个完整的URL。}\\

\noindent{\color{red}{urllib.request.url2pathname(path):}}
\par{转换the path component path from a percent-encoded URL to the local syntax for a path.}\\

\noindent{\color{red}{urllib.request.getproxies():}}
\par{返回一个代理服务器调用的map的字典。}\\

下面的这些类也是提供的:
\noindent{\color{red}{class urllib.request.Request(url, data=None, headers={},\\
origin\_req\_host=None, unverifiable=False, method=None):}}
\par{未知}\\

\noindent{\color{red}{class urllib.request.OpenerDirector:}}
\par{未知}\\

\noindent{\color{red}{class urllib.request.BaseHandler:}}
\par{未知}\\

\noindent{\color{red}{class urllib.request.HTTPDefaultErrorHandler:}}
\par{未知}\\

\noindent{\color{red}{class url.request.HTTPRedirecHandler:}}
\par{未知}\\

\noindent{\color{red}{class urllib.request.HTTPCookieProcessor(cookiejar=None):}}
\par{未知}\\

\noindent{\color{red}{class urllib.request.ProxyHandler(proxies=None):}}
\par{未知}\\

\noindent{\color{red}{class urllib.request.HTTPPasswordMgr:}}
\par{未知}\\

\noindent{\color{red}{class urllib.request.HTTPPasswordMgrWithDefaultReal:}}
\par{未知}\\

\noindent{\color{red}{class urllib.request.AbstractBasicAuthHandler(password\_mgr=None):}}
\par{未知}\\

\noindent{\color{red}{class urllib.request.HTTPBasicAuthHandler(password\_mgr=None):}}
\par{未知}\\

\noindent{\color{red}{class urllib.request.ProxyBasicAuthHandler(password\_mgr=None):}}
\par{未知}\\

\noindent{\color{red}{class urllib.request.AbstractDigestAuthHandler(password\_mgt=None):}}
\par{未知}\\

\noindent{\color{red}{class urllib.request.HTTPDigestAuthHandler(password\_mgr=None):}}
\par{未知}\\

\noindent{\color{red}{class urllib.request.ProxyDigestAuthHandler(password\_mgt=None):}}
\par{未知}\\

\noindent{\color{red}{class urllib.request.HTTPHandler:}}
\par{未知}\\

\noindent{\color{red}{class urllib.request.HTTPSHandler(debuglevel=0, context=None,\\
 check\_hostname=None):}}
\par{未知}\\

\noindent{\color{red}{class urllib.request.FileHandler:}}
\par{未知}\\

\noindent{\color{red}{class urllib.request.DataHandler:}}
\par{未知}\\

\noindent{\color{red}{class urllib.request.FTPHandler:}}
\par{未知}\\

\noindent{\color{red}{class urllib.request.CacheFTPHandler:}}
\par{未知}\\

\noindent{\color{red}{class urllib.request.UnknownHandler:}}
\par{未知}\\

\noindent{\color{red}{class urllib.request.HTTPErrorProcessor:}}
\par{未知}\\

\subsection{Request Objects}
下面的方法描述了Request的共有方法,因此下面的所有的方法都可以在继承类中进行重载。\\

\noindent{\color{red}{Request.full\_url:}}
\par{未知}\\

\noindent{\color{red}{Request.type:}}
\par{未知}\\

\noindent{\color{red}{Request.host:}}
\par{未知}\\

\noindent{\color{red}{Request.origin\_req\_host:}}
\par{未知}\\

\noindent{\color{red}{Request.selector:}}
\par{未知}\\

\noindent{\color{red}{Request.data:}}
\par{未知}\\

\noindent{\color{red}{Request.unverifiable:}}
\par{未知}\\

\noindent{\color{red}{Request.method:}}
\par{未知}\\

\noindent{\color{red}{Request.get\_method():}}
\par{未知}\\

\noindent{\color{red}{Request.add\_header(key, val):}}
\par{未知}\\

\noindent{\color{red}{Request.add\_unredirected\_header(key, header):}}
\par{未知}\\

\noindent{\color{red}{Request.has\_header(header):}}
\par{未知}\\

\noindent{\color{red}{Request.remove\_header(header):}}
\par{未知}\\

\noindent{\color{red}{Request.get\_full\_url():}}
\par{未知}\\

\noindent{\color{red}{Request.set\_proxy(host, type):}}
\par{未知}\\

\noindent{\color{red}{Request.get\_header(header\_name, default=None):}}
\par{未知}\\

\noindent{\color{red}{Request.header\_items():}}
\par{未知}\\

\subsection{OPenerDirector Objects}
OpenerDirector instance有下面的方法。

\noindent{\color{red}{OpenerDirector.add\_Handler(Handler):}}
\par{未知}\\

\noindent{\color{red}{OpenerDirector.open(url, data=None[, timeout]):}}
\par{未知}\\

\noindent{\color{red}{OpenerDirector.error(proto, *args):}}
\par{未知}\\

\subsection{BaseHandler Objects}
Basehandler objects provide a couple of methods that are directly useful, and others that are
meant to be used by derived classes. These are intended for direct use:

\noindent{\color{red}{BaseHandler.add\_parent(director):}}
\par{未知}\\

\noindent{\color{red}{BaseHandler.close():}}
\par{未知}\\





\section{urllib.response}
the urllib.response module defines functions and classes which define a minimal file like
interface, including read() and readline(). The typical reponse object is an addinfourl instance,
 which defines an info() method and that returns headers and a geturl() method that returns the
url. Functions defined by this module are used internally by the urllib.request module.




\section{urllib.parse}
This module defines a standard interface to break Uniform Response Locator(URL) strings up in
components (addressing scheme, network location, path etc.) to combine the components back into
a URL string, and to convert a \textquotedblleft{relative URL}\textquotedblright{} to an absolute
URL given a \textquotedblleft{base URL}\textquotedblright{}.\\

The module has been designed to match the internet RFC on Relative Uniform Resource Locator. It
supports the following scheme: file, ftp, gopher, hdl, http, https, imap, mailto, mms, news, nntp,
prospero, rsync, rtsp, rtspu, sftp, shttp, sip, sips, snews, svn, svn+ssh, telnet, wais, ws, wss.





\section{urllib.error}





\section{urllib.robotparser}





\section{http}





\section{http.client}






\section{ftplib}






\section{poplib}







\section{imaplib}






\section{nntplib}






\section{smtplib}






\section{smtpd}






\section{telnetlib}






\section{uuid}






\section{socketserver}






\section{http.server}






\section{http.cookies}






\section{http.cookiejar}






\section{xmlrpc}







\section{xmlrpc.client}






\section{xmlrpc.server}






\section{ipaddress}


%%% Local Variables:
%%% mode: latex
%%% TeX-master: t
%%% End:
