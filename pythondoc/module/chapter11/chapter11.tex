%#-*- coding:utf-8 -*-
\chapter{文件和目录}
\section{pathlib}






\section{os.path}
\noindent{\color{red}{os.path.abspath(path):}}
\par{返回路径名path的规范化的绝对路径,在大多数的9平台上,该函数等同于normpath(),即normpath(join(os.getcwd(), path))。}\\

\noindent{\color{red}{os.path.basename(path):}}
\par{返回路径名path的主文档名。在path上调用split()函数,返回的二元组中的第二个元素就是主文档名。这和Unix的basename不同,
  对于\textquoteleft{/foo/bar/}\textquoteright{},basename返回\textquoteleft{bar}\textquoteright{},而basename()函数返回
  空字符串。}\\

\noindent{\color{red}{os.path.commonpath(paths):}}
\par{未知}\\

\noindent{\color{red}{os.path.commonprefix(list):}}
\par{返回list中所有路径的最长通用前缀(逐字符的进行比较)。如果list为空,返回空字符串(\textquoteleft{}\textquoteright{})。
  因为是逐字符比较,所以可能会返回无效的路径。}\\

\noindent{\color{red}{os.path.dirname(path):}}
\par{返回路径名path的目录名。在path上调用split(),返回的二元组中的第一个元素就是目录名。}\\

\noindent{\color{red}{os.path.exists(path):}}
\par{如果path引用一个存在的路径,返回True。如果path引用一个断开的符号链接,返回False。在某些平台上,尽管path物理存在,但
  是由于没有执行os.stat()的权限,该函数也会返回False。}\\

\noindent{\color{red}{os.path.lexists(path):}}
\par{如果path引用一个存在的路径,返回True.对于path引用一个断开的符号链接,也返回True。等同于缺少os.lstat()的平台上的exists()。}\\

\noindent{\color{red}{os.path.expanduser(path):}}
\par{在Unix和Windows平台上,返回参数,参数中开头的$\sim$或者$\sim$user被替换成user的主家目录。在Unix上,开头的$\sim$被替
  换成环境变量HOME,如果它被设置的话;否则,通过内建模块pwd在密码目录查询当前用户的家目录。如果开头是$\sim$user,则直接
  在密码目录中查询(user的家目录)。}
\par{在Windows上,将使用HOME和USERPROFILE,如果它们被设置的话;否则使用HOMEPATH和HOMEDRIVE的组合。如果开头是$\sim$user,首先按上述方式得到user路径,然后移除最后的目录部分。如果扩展失败或者参数path不是以$\sim$打头,则直接返回参数(path)。
}\\

\noindent{\color{red}{os.path.expandvars(path):}}
\par{返回参数,其中的环境变量被扩展。参数中形如\$name或者\${name}的部分被替换成环境变量name的值。如果格式不正确或者引用了不存在的变量,则不进行替换(或者扩展)。在Windows上, 除了\$name和\${name}之外还支持如\%name\%这样的扩展。
}\\

\noindent{\color{red}{os.path.getatime(path):}}
\par{返回path的最后访问时间。返回的是从Unix纪元开始的跳秒数(epoch,Unix纪元,即GMT 1970-01-01 00:00:00)(参见time模块)。如果文件不存在或者不可访问,返回os.error。}\\

\noindent{\color{red}{os.path.getmtime(path):}}
\par{返回path的最后修改时间。返回的是从Unix纪元开始的跳秒数(参见time模块)。如果文件不存在或者不可访问,返回os.error。}\\

\noindent{\color{red}{os.path.getctime(path):}}
\par{返回系统的ctime,在Unix这样的系统上,它是文件元数据最后修改时间(或者可以说是文件状态最后修改时间);在Windows这样  的系统上,它是path的创建时间。返回的是从Unix纪元开始的跳秒数(参见time模块)。如果文件不存在或者不可访问,返回os.error。}\\

\noindent{\color{red}{os.path.getsize(path):}}
\par{返回path的大小,以字节为单位。如果文件不存在或者不可访问,返回os.error。}\\

\noindent{\color{red}{os.path.isabs(path):}}
\par{如果path是绝对路径名,返回True。在Unix上,这表示路径以/开始;在Windows上,这表示路径以\textbackslash{}开始(在去掉可能的盘符后,如C:)。}\\

\noindent{\color{red}{os.path.isfile(path):}}
\par{如果path是一个存在的普通文件,返回True。它会跟随符号链接,所以对相同的路径,islink()和isfile()可以同时为真。}\\

\noindent{\color{red}{os.path.isdir(path):}}
\par{如果path是一个存在的目录,返回True。它会跟随符号链接,所以对于相同的路径,islink()和isdir()可以同时为真。}\\

\noindent{\color{red}{os.path.islink(path):}}
\par{如果path引用的目录条目是个符号链接,返回True。如果Python运行期不支持符号链接,则总是返回False。}\\

\noindent{\color{red}{os.path.ismount(path):}}
\par{如果路径名path是一个mount point(挂载点),返回True。挂载点是文件系统中的一点,不同的文件系统在此被挂载。它检查path
的父目录path/..和path是否在不同的设备上,或者检查path/..和path是否指向相同设备的相同i-node,——这可以检查出Unix和POSIX变种下的挂载点。}\\

\noindent{\color{red}{os.path.join(path, *paths):}}
\par{将一个或多个路径正确地连接起来。如果任何一个参数是绝对路径,那之前的参数就会被丢弃,然后连接继续。}\\

\noindent{\color{red}{os.path.normcase(path):}}
\par{未知}\\

\noindent{\color{red}{os.path.normpath(path):}}
\par{未知}\\

\noindent{\color{red}{os.path.realpath(path):}}
\par{未知}\\



\section{fileinput}





\section{stat}





\section{filecmp}





\section{tempfile}





\section{glob}





\section{fnmatch}





\section{linecache}





\section{shutil}





\section{macpath}






%%% Local Variables:
%%% mode: latex
%%% TeX-master: t
%%% End:
