%#-*- coding:utf-8 -*-
\chapter{引言}
这本书是关于用计算机解决统计物理学中问题的,特别是用蒙特卡罗方法。
\section{统计力学}
统计力学主要考虑用来计算condensed matter系统的性质。通常这些系统由众多的部分组成,典型的如原子或者分子。这些部分通常相同
且它们遵循着相对简单的运动方程,以至于可以直接的写出数学表达式。但是这些运动方程的数量是众多的,以至于我们无法得到系统精
确的数学解。对于这些\textquotedblleft{大系统}\textquotedblright{}而言,解这些系统的Hamilton's方程是不可能的,因为有太多
的等式。然而当我们观察系统的宏观特性时,这些特性是如此的well-behaved和可预测的。Clearly, there is something special
about the behaviour of the solutions of these many equations that \textquotedblleft{average out}\textquotedblright{} to
give us a pridictable behaviour for the entire system。例如气体的压强和体积就遵循简单的规律。统计物理学将解大量的运动方
程放在一边,而是从系统的整体特性出发,考虑系统处在某个状态的概率---系统可能处在这个状态或者另一个状态,并且有其对应的压
强或者其它的物理量。我们发现统计物理学的概率描述是如此的有用,because we usually find that for large systems the range
of behaviours of the system that are anything more than phenomenally unlikely is very small; all the reasonably probable
behaviours fall into a narrow range,允许我们以极高的置信度去描述系统。

在这本书的剩下部分,我们主要探讨系统由Hamiltonian函数描述的能量确定的状态。我们主要探讨分立的状态\textemdash{}每个状态有其自己的能量$E_{0},E_{1},E_{2}$\dots{},可能没有上限。同样的统计力学和模特卡罗方法同样的适用于连续的状态。

如果我们拥有系统的Hamiltonian量,系统的能量就可以确定下来。然而系统通常并不是独立的,而是和其它的thermal reservoir进行这
热交换。(参看朗道的统计物理学)

假设我们的系统处于状态$\mu$。我们用$R(\mu\rightarrow{}\nu)dt$表示系统在$dt$时刻后处在$\nu$的概率。
$R(\mu\rightarrow{}\nu)$是系统的{\textbf{transition rate}}对于从$\mu$到$\nu$的转变。transiton rate通常是不含时的。对于初始的状态$\mu$经过一段时间后,系统可能处在大量的其它的状态之一,而这就是我们概率处理问题热力学的原因。我们定义一系列的权重
$\omega_{\mu}(t)$来代表系统在时刻$t$处在状态$\mu$的概率。Statistical mechanics deals with these weights, and they
represent out entire knowledge about the state of the system.对于$\omega_{\mu}(t)$的演化,我们写出一个关于
$R(\mu\rightarrow{}\nu)$的\textbf{master equation}:
\begin{equation}
  \label{eq:1}
  \frac{dw_{\mu}}{dt} = \sum_{\nu}\left[\omega_{\nu}(t)R(\nu\rightarrow{}\mu)-\omega_{\mu}(t)R(\mu\rightarrow{}\nu)\right].
\end{equation}
右边第一项是转变到状态$\mu$的概率,第二项是从状态$\mu$到其它状态的概率。概率$\omega_{\mu}(t)$遵循求和规则
\begin{equation}
  \label{eq:2}
  \sum_{\mu}\omega_{\mu}(t) = 1
\end{equation}
对所有的$t$。
\par
如何将权重$\omega_{\mu}$和系统的宏观状态联系起来呢?例如我们对物理量$Q$感兴趣,对每一个状态$\mu$,其对应的物理量为
$Q_{\mu}$,我们可以定义$Q$的期望值在时刻$t$对于我们的系统:
\begin{equation}
  \label{eq:3}
  \left<Q\right> = \sum_{\mu}Q_{\mu}\omega_{\mu}(t).
\end{equation}
很明显该期望值和我们在实验中测量到的物理量$Q$有很大的关系.例如,系统处在状态$\tau$,则期望值$\left[Q\right].$对应的测量值
即为$Q_{\tau}$,又如系统等可能的处在三个状态,且处在其它状态的概率为0.则期望值等于三个状态$Q$的平均值.然而对于平均值的理解,存
在两种不同的观点。\textbf{第一种},我们将要考虑的系统进行大量的复制,瞬间的测量这些众多的系统,然后给出我们要考虑的系统平
均值。然而实际上,我们只有一个这样的系统,无法在实验上进行复制。\textbf{第二种},我们对要考虑的系统进行大量反复的测量,
且出现\textquotedblleft{representative selection of the states}\textquotedblright{}。这
有点类似于(\ref{eq:3})的处理。只要重复测量的次数足够的多,我们就能更好的增加测量值和期望值之间近似度。
\par
然而这里有存在两个问题,\textbf{第一},什么才叫\textquotedblleft{representative selection of the
  states}\textquotedblright.可能在我们的测量过程中,系统只在某两个状态之间进行转变,而我们的测量刚好都在一个状态。而又或
者系统在快速的进行改变,并没有达到平衡的状态,这是一个很重要的问题。\textbf{第二},另一个问题就是权重参数
$\omega_{\mu}(t)$是含时的,对系统的测量可能会影响到$\omega_{\mu}(t)$,从而使我们的表达式无效。这些都是在实验和模拟中存在
的非平衡态系统问题,而这正是这本书第二部分要处理的问题。对于平衡系统,在我们以后的讨论中,权重参数都是不随时间演化的。
\par
尽管存在这么多的问题,时间平均的解释在实验和模拟上都被广泛的应用。








\section{平衡态}
让我们再考虑\textbf{master equation}(\ref{eq:1})。如果等式的右边两部分可以抵消掉,即$d\omega_{\mu}/dt$等于0.也就是权
重参数$\omega$将是常数。这时系统达到平衡态,而本书的大部分就是处理平衡态,在这部分我们主要引入一些重要的统计概念。
\par
在master equation中的转变概率$R(\mu-\nu)$do not just take any values. They take particular values which arise out of the
thermal nature of the interaction between the system and the thermal reservoir. In the later chapter of this book we
will have to choose values for these rates when we simulate thermal systems in out Monte Carlo calculations with the
thermal reservoir correctly.重要的是,对于系统我们知道一些先验的权重值$\omega_{\mu}$。我们将这些平衡值称为
\textbf{equilibrium occupation probabilities},并表示为
\begin{equation}
  \label{eq:4}
  p_{\mu} = \lim_{t\rightarrow\infty}\omega_{\mu}(t).
\end{equation}
吉布斯(1902)证明对于一个处在温度T下的系统,equilibrium occupation probabilities是
\begin{equation}
  \label{eq:5}
  p_{\mu} = \frac{1}{Z}e^{E_{\mu}/kT}.
\end{equation}
其中$E_\mu$是状态$\mu$下的能量,k是玻尔兹曼常数,其值为1.38$\texttimes$$10^{-23}JK^{-1}$,习惯上我们用$\beta$来表示
$(KT)^{-1}$,Z是归一化常数
\begin{equation}
  \label{eq:6}
  Z = \sum_{\mu}e^{-E_{\mu}/kT} = \sum_{\mu}e^{-{\beta}E_{\mu}}
\end{equation}
Z就是配分函数,it figures a lot more heavily in the mathematical development of statistical mechanics than a mere
normalizing constant might be expected to.事实上,Z作为温度T和其它的影响系统的参数,几乎可以表征系统的任何物理量。概率分
布(\ref{eq:5})即玻尔兹曼分布。In our treatment, we will take Equation (\ref{eq:5})作为我们进一步讨论的出发点。
%%% Lo$cal Variables:
%%% mode: latex
%%% TeX-master: t
%%% End:
