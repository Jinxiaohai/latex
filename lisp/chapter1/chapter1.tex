%#-*- coding:utf-8 -*-
\chapter{列表处理}
\section{lisp列表}
lisp通过把列表放置在括号之间来处理列表的编程语言,括号标记了列表的边界,对于列表中的元素采用
空格进行分割。

\section{运行一个程序}
lisp中的一个列表,任何一个列表,都是一个正在准备运行的程序。如果你运行它,计算机
将完成三件事情:
\begin{enumerate}
\item 只返回列表本身。
\item 告诉你一个出错的消息。
\item 将列表中的第一个符号当作一个命令,然后执行这个命令。
\end{enumerate}
单引号,也就是列表前面的那个单引号,被称作引用。当单引号位于一个列表前面时,它告诉lisp不要对这个
列表进行任何操作,而仅仅是按照其原样。但是,如果一个列表前面没有引号,这个列表的第一个符号就很特
别了:它是一条计算机要执行的命令(在lisp中,这些命令被称作函数)。

\section{Lisp解释器}
如果lisp解释器正在寻找的函数不是一个特殊表,而是一个列表的一部分,则lisp解释器首先查看这个列表中
是否有另外一个列表。如果有一个内部列表,lisp解释器首先解释将如何的处理内部的列表,然后再处理外层的
列表。如果还有一个列表嵌入在内层列表中,则解释器首先解释那个列表,然后逐一往外解释。它总是首先处
理最内层的列表。

\section{求值}
当lisp解释器处理一个表达式时,这个动作被称为$\textquoteleft$求值$\textquoteright$。完成求值之后,lisp
解释器几乎总要返回一个值,这个值是计算机执行它时在函数定义中找到的指令的结果。在解释器返回一个值
的时候,它还可以做些其它的什么事情,例如移动光标或者拷贝一个文件,这种动作称为附带效果。

\section{变量}
在lisp中,可以将一个值赋给一个符号。任何值都可以赋给一个符号,也就是将变量和值进行绑定。

\section{concat函数}
concat函数将两个或者更多的字符串连接起来,产生一个新的字符串。

\section{message函数}
像+函数一样,message函数的参数数目是可以变化的。它被用于给用户发送信息。

\section{给一个变量赋值}
有几种方法给一个变量赋值。其中一种是使用set函数或者使用setq函数,另外一种是使用let函数。

\subsection{使用set函数}
为了将符号flowers的值设为列表$\textquoteright$(rose violet daisy buttercup),将光标置于下面的
表达式之后并键入C-x C-e来对表达式求值:\\
(set $\textquoteright$flowers $\textquoteright$(rose violet daisy buttercup))\\
列表(rose violet daisy buttercup)将出现在回显区中。这是set函数返回的值。作为一个附带效果,符号
flowers将被绑定到一个列表上,也就是列表作为值被赋给可以被当做变量的符号flowers。对set表达式求值
以后,能对符号flowsers求值,它将返回刚刚设定的值。下面就是这个符号。将光标置于它后面并键入C-x C-e:\\
flowers\\
当对flowers求值时,列表(rose violet daisy buttercup)显示在回显区。\\
同样要注意,当使用set函数时,需要将set函数的两个参量都用引号限定起来,除非你希望它们被求值。

\subsection{使用setq函数}
实际上,人们几乎总是将set函数的第一个参量用单引号标出。set函数和其第一个带引号的参量的组合是
如此的常用,以至于它有一个自己的名字:setq特殊表函数。这个特殊表就像set函数一样,不同之处在于
其第一个参量自动地带上单引号。因此,不必自己键入单引号了。同样,另一个方便之处在于,setq函数
允许在一个表达式中将几个变量设置成不同的值。


%%% Local Variables:
%%% mode: latex
%%% TeX-master: t
%%% End:
