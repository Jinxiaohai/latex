\documentclass[11pt,a4paper,openany]{article}
\usepackage{amsmath}             %%%%多种的公式环境和数学命令
\usepackage{amssymb}             %%%%数学符号生成命令
\usepackage{array}               %%%%数组和表格
\usepackage{booktabs}            %%%%水平的表格线
\usepackage{calc}                %%%%四则运算
\usepackage{caption}             %%%%插图和表格
% \usepackage{ctex}                %%%%中文字体
% \usepackage{ctexcap}             %%%%中文字体和标题
\usepackage{fancyhdr}            %%%%页眉页脚设置
\usepackage{graphicx}            %%%%插图
\usepackage{geometry}            %%%%版面尺寸控制
\geometry{left=2cm, right=2cm, top=2cm, bottom=2cm, head=2cm, foot=1cm}
% head=?cm, headmap=?cm
\usepackage{hyperref}            %%%%超链接
\usepackage{ifthen}              %%%%条件
\usepackage{longtable}           %%%%跨页表格
\usepackage{multicol}            %%%%多栏
\usepackage{ntheorem}            %%%%定理设置
\usepackage{paralist}            %%%%列表
\usepackage{tabularx}            %%%%表格的列宽
\usepackage{titlesec}            %%%%章节标题
\usepackage{fancyvrb}            %%%%抄录
\usepackage{fontspec}            %%%%字体
\usepackage{titletoc}            %%%%目录格式
\usepackage{xcolor}              %%%%颜色处理
\usepackage{xeCJK}               %%%%中日朝文字处理

\newcommand{\mycmdA}{ }
\newcommand{\mycmdB}[1]{{\heiti #1}}
\newcommand{\mycmdC}[2]{$#1_1,#1_2,\dots,#1_#2$}

\title{\vspace{-20mm}\textbf{\Large Omega and Phi production in a multiphase transport model with enhanced local parton
density fluctuation scenario}}
\author{Xiaohai Jin(金小海) \and J. H. Chen \and Y. G. Ma}
\date{Shanghai Institute of Applied Physics, CAS, Shanghai}

\begin{document}
\maketitle

\begin{abstract}
  Searching for QCD critical point and mapping the QCD phase diagram are major science goals of the Beam Energy Scan program in Heavy-Ion Collisions. Many exciting results have been published in the past decades and deepen our understanding on the QCD phase transition, such as the non-monotonic of net-proton direct flow, the net-baryon kurtosis. On the other hand, multi-strange hadron such as Omega and Phi are important probes for the search of the QCD phase boundaries. The Omega and Phi are expected to have relatively small hadronic interaction cross sections. Therefore, they can carry the information directly from the chemical freeze-out stage with little or no distortion due to hadronic rescattering. As a result, the production of the Omega and Phi particle offers a unique advantage in probing the transition from partonic to hadronic dynamics.  In this talk, we study the Omega and Phi production in a multiphase transport model with employment enhanced local parton density fluctuation scenario. Our calculations describe the pt spectrum of experiment data well and predict an enhanced production of Omega and Phi near the QCD phase boundary. In particularly, we find that the Baryon/Meson ratio is more sensitive to the local density fluctuation strength. The ratio will increased significantly in comparison with original AMPT model calculation. We also study the flow harmonic of multistrange hadron in response to the local density fluctuation. Physics implication will be discussed. 
\end{abstract}

\section{Introduction}


\end{document}

%%% Local Variables:
%%% mode: latex
%%% TeX-master: t
%%% End:
