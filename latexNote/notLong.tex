%#-*- coding:utf-8 -*-
\documentclass[11pt,a4paper,UTF8,openany]{ctexbook}

\usepackage{ctex}
\usepackage{color}
\usepackage{xcolor}


\author{xiaohai}
\date{2017/04/06}
\title{xiaohai}

\begin{document}

\maketitle
\tableofcontents

\chapter{a not long file}
\section{开始}
多个连续的空白字符等于一个空白的字符,句首的空白距离一般会忽略,单个的空行
等于一个空白距离,两个空白文本之间的空白标志着上段的结束和下段的开始,多个
的空白行等于一个空白行。

\noindent 重启一段的命令{} : {\color{red}$\backslash$par}\newline
颜色需要使用的包 : {\color{red}{xcolor}}\newline
下面是一些常见的特殊字符以及对应的命令:\newline
\# : {\color{red}$\backslash\#$}\newline
\$ : {\color{red}$\backslash\$$}\newline
\% : {\color{red}$\backslash\%$}\newline
\^{} : {\color{red}$\backslash$\^{}$\{\}$}\newline
\~{} : {\color{red}$\backslash$~$\{\}$}\newline
反斜线不等通过$\backslash\backslash$得到,该命令是用来换行的.
需要使用$\backslash$\$backslash\$得到.\newline

命令的两种形式:\newline
1.以一个反斜线开始,命令只有字母组成。命令后的空格符、数字、或任何非字母的字符
都标志着命令的结束。\newline
2.以一个反斜线和非字母的字符组成。\newline
latex使用\%进行注释,多行注释使用verbatim宏包提供的comment环境。\newline

\section{进入正题}
每个源文件都以$\backslash$doucumentclass\{...\}开始。\newline
导入宏包使用$\backslash$usepackage\{...\}。\newline
文档的开始使用$\backslash$begin\{doucument\}。\newline
文档的结束使用$\backslash$end\{document\}。\newline

\section{文档的布局}



%_______________________________________________________________________
\chapter{数学公式的处理}
\section{小项的处理}
\begin{enumerate}
\item 根号$\sqrt{2}$ : \begin{verbatim}$\sqrt{2}$\end{verbatim}
\item 上标$x^{2}$ : \begin{verbatim}$x^{2}$\end{verbatim}
\item 下标$x_{2}$ : \begin{verbatim}$x_{2}$\end{verbatim}
\item 短横线\textrm{-} : \begin{verbatim}\textrm{-}\end{verbatim}
\item 来个复杂的$^{16}O^{2-}_{32}$ : \begin{verbatim}$^{16}O^{2-}_{32}$\end{verbatim}
\item 偏导下标$\partial f_{\mbox{\tiny3}}$ : 如果使用文字作为下标,要将文字模式里
\begin{verbatim}$\partial f_{\mbox{\tiny3}}$\end{verbatim}
\item 分式$\frac{num}{div}$ : \begin{verbatim}$\frac{分子}{分母}$\end{verbatim}
\item n次根号$\sqrt[n]{func}$ : \begin{verbatim}$\sqrt[n]{表达式}$\end{verbatim}
\item 求和$\sum_{k=1}^n(func)$ : \begin{verbatim}$\sum_{k=1\}^n(表达式)$\end{verbatim}

\end{enumerate}



%_____________________________________________________________________
\chapter{各种括号大总结}
\section{我是括号}
\noindent 圆括号$\left(\frac{a}{b}\right)$ : 
\begin{verbatim}\left(\frac{a}{b}\right)\end{verbatim}
方括号$\left[\frac{a}{b}\right]$ : 
\begin{verbatim}\left[\frac{a}{b}\right]\end{verbatim}
大括号$\left\{\frac{a}{b}\right\}$ :
\begin{verbatim}\left\{\frac{a}{b}\right\}\end{verbatim}
尖括号$\left\langle\frac{a}{b}\right\rangle$ :
\begin{verbatim}\left\langle\frac{a}{b}\right\rangle\end{verbatim}
绝对值$\left|\frac{a}{b}\right|$ :
\begin{verbatim}\left|\frac{a}{b}\right|\end{verbatim}
双竖线$\left\|\frac{a}{b}\right\|$ :
\begin{verbatim}\left\|\frac{a}{b}\right\|\end{verbatim}
混合括号$\left[a,b\right)$ :
\begin{verbatim}\left[a,b\right)\end{verbatim}
混合括号2$\left\langle\psi\right|$ :
\begin{verbatim}\left\langle\psi\right|\end{verbatim}

\end{document}
