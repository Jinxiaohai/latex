%#-*- coding:utf-8 -*-
\documentclass[11pt,a4paper,UTF8]{ctexbook}

\usepackage{ctex}
\usepackage{color}
\usepackage{xcolor}


\author{xiaohai}
\date{2017/04/06}
\title{xiaohai}

\begin{document}

\maketitle
\tableofcontents

\chapter{a not long file}
\section{开始}
多个连续的空白字符等于一个空白的字符,句首的空白距离一般会忽略,单个的空行
等于一个空白距离,两个空白文本之间的空白标志着上段的结束和下段的开始,多个
的空白行等于一个空白行。

\noindent 重启一段的命令{} : {\color{red}$\backslash$par}\newline
颜色需要使用的包 : {\color{red}{xcolor}}\newline
下面是一些常见的特殊字符以及对应的命令:\newline
\# : {\color{red}$\backslash\#$}\newline
\$ : {\color{red}$\backslash\$$}\newline
\% : {\color{red}$\backslash\%$}\newline
\^{} : {\color{red}$\backslash$\^{}$\{\}$}\newline
\~{} : {\color{red}$\backslash$~$\{\}$}\newline
反斜线不等通过$\backslash\backslash$得到,该命令是用来换行的.
需要使用$\backslash$\$backslash\$得到.\newline

命令的两种形式:\newline
1.以一个反斜线开始,命令只有字母组成。命令后的空格符、数字、或任何非字母的字符
都标志着命令的结束。\newline
2.以一个反斜线和非字母的字符组成。\newline
latex使用\%进行注释,多行注释使用verbatim宏包提供的comment环境。\newline

\section{进入正题}
每个源文件都以$\backslash$doucumentclass\{...\}开始。\newline
导入宏包使用$\backslash$usepackage\{...\}。\newline
文档的开始使用$\backslash$begin\{doucument\}。\newline
文档的结束使用$\backslash$end\{document\}。\newline

\section{文档的布局}


\end{document}
